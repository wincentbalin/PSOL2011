\documentclass[12pt,a4paper,titlepage]{article}

% Input encoding
\usepackage[utf8x]{inputenc}
\usepackage[T1,T2A]{fontenc}
\usepackage[english,russian,german]{babel}

\usepackage{dramatist}

\author{Lev Mirkin}
\title{Purimspiel Oldenburg 2011}

\begin{document}
\maketitle


% Russian improvements
\renewcommand{\casttitlename}{Действующие лица}

% Make scene in the kind of "Kartina 1"
\renewcommand{\scenename}{Картина}
\renewcommand{\thescene}{\arabic{scene}}
\renewcommand{\printscenenum}{\scenenumfont \thescene}

% Make titles and speakers bold
\renewcommand{\casttitlefont}{\bfseries\scshape\large} % Make "dramatis personae" actually smaller
\renewcommand{\scenenamefont}{\bfseries\scshape\large}
\renewcommand{\speaksfont}{\bfseries\scshape}


% Insert distance before a scene
\setlength{\beforesceneskip}{30pt}


% Enlarge default font
\fontsize{14}{17} \selectfont


\Character[Рассказчик, он же переводчик]{Erzähler}{ue}
\Character[Мордехай]{Мордехай}{m}
\Character[Ахашверош]{Ахашверош}{ah}
\Character[Вашти]{Вашти}{v}
\Character[Аман]{Аман}{am}
\Character[Эстер]{Эстер}{e}

\DramPer



\addtocounter{scene}{-1}
\Scene{ -- Вступление}

\StageDir{Музыка танцевальная еврейская.\par Nach der ersten Strophe kommt der \ue vor die Bühne}

\begin{drama}

\uespeaks
Wir sind heute zusammen gekommen, um gemeinsam das Purim-Fest zu feiern.
Ein Fest des Sieges des Jüdischen Volkes über die bösen Mächte. Ich möchte
euch diese Geschichte gleich erzählen, und die Puppen des Figurentheaters Lappanoptikum
werden mir bei dieser Gelegenheit helfen.


\scene

\StageDir{Занавес открывается. На сцену выходит \m.}

\mspeaks
Schalom, ich begrüße die ganze Welt, und das Jüdische Volk und Sie alle, liebe Zuschauer!
\direct{dem Erzähler} Was für ein schöner und friedlicher Abend, Herr Erzähler.
Aber, ich höre gerade einige verdächtige Stimmen. Wer kann es sein?

\StageDir{\m и \ue прячутся.\\Открывается окошко. В нём две головы.}

\speaker{Головы} \direct{Поют на мотив \textsc{Бяки-Буки}}
\begin{verse}
Говорят мы террористы\\
И не любит нас народ,\\
Не танцуем танцы твисты\\
А совсем наоборот.\\
\end{verse}

\begin{verse}
Ой-ля-ля ой-ля-ля,\\
Завтра свергнем мы царя,\\
Ой-ля-ля ой-ля-ля,\\
Эх, ма!\\
\end{verse}

\uespeaks
Schalom, Herr Mordechai. Es sind die Generäle. In den friedlichen Zeiten bekommen
sie Langeweile, und fangen Blödsinn an. Sie wollen unseren Herrscher, Achaschwerosch, entthronen.

\mspeaks \direct{dem Erzähler}
Oh, dann muss ich mich beeilen und darüber dem Achaschwerosch berichten.
Die Geschichte lehrt uns, dass in den unsicheren Zeiten immer die Juden bestraft wurden.

\StageDir{\m уходит.}


\scene

\StageDir{Выходит \ah.}

\ahspeaks \direct{dem Erzähler}
A-ha, das Volk hat sich gerade versammelt, wie ich sehe. Gut. Dann lese ich
meine Ansprache an die Versammlung der Nationen vor. Du –- bist du Dolmetscher?
Gut, du wirst sie übersetzen. \direct{Zum begleitenden Musiker} Maestro-e-musica, mach Musik!

\ahspeaks \direct{Поёт на мотив \textsc{Бяки-Буки}}
\begin{verse}
Говорят: цари -- плохие,\\
Как же терпит их народ?\\
Демократов расплодили,\\
А у нас -- наоборот!\\
\end{verse}

\begin{verse}
\textit{Припев}\\
Ой-ля-ля ой-ля-ля,\\
Персы любят короля,\\
Ой-ля-ля ой-ля-ля,\\
Эх, ма!\\
\end{verse}

\begin{verse}
С добрым утром, персияне,\\
Мой народ, моя родня!\\
Я без мысли об Иране\\
Не могу прожить и дня.\\
\end{verse}

\begin{verse}
\textit{Припев}\\
\end{verse}

\begin{verse}
Ночью встану у окна я\\
И стою совсем без сна:\\
Как там Персия родная,\\
Как там бедная она?\\
\end{verse}

\begin{verse}
\textit{Припев}\\
\end{verse}

\uespeaks
In dieser Ansprache besingt unser demokratielenkender Herrscher seine Sorge
um sein Reich. Er betont, dass er vor Kummer fast keinen Appetit mehr hat –-
er kann den Kaviar nicht herunterschlucken! Und schlafen nach dem üppigen Essen
kann er auch nicht gut; vor Kummer, versteht sich.


\scene

\StageDir{На сцену выходит \m.}

\mspeaks \direct{Поёт на мотив \textsc{Какая чудная земля}}
\begin{verse}
Какая чудная земля\\
Родная Персия твоя\\
И счастливо живут в ней все народы.\\
Но в ней я заговор раскрыл\\
К тебе я сразу поспешил,\\
Чтоб не терять иллюзию свободы.\\
\end{verse}

\uespeaks
Mordechai erstattete dem Herrscher genauestens Bericht über den Komplott der Generäle.

\ahspeaks \direct{Кладёт руку Мордехаю на плечо.\\Поёт на мотив \textsc{Какая чудная земля}}
\begin{verse}
Ты молодец, что прибежал\\
И мне об этом рассказал.\\
Казнить их всех! Предатели! Путчисты!\\
\end{verse}

\begin{verse}
Тебя я позже награжу,\\
Сейчас на пир я поспешу.\\
Лехаим всем! Убиты террористы!\\
\end{verse}

\StageDir{Вылетают чалмы казнённых.}

\uespeaks
Die Generäle wurden hingerichtet. Der Zar hat dem Mordechai versprochen ihn zu belohnen,
aber später. Er selbst ist zu Feierlichkeiten gegangen.

\StageDir{\ah уходит.}


\scene

\mspeaks \direct{Поёт на мотив \textsc{Мой костёр в тумане светит}}
\begin{verse}
Мой народ по миру бродит,\\
На земле чужой живёт,\\
Счастья там он не находит,\\
Всё о родине поёт.\\
\end{verse}

\begin{verse}
Я принёс царю прошенье:\\
Отпусти ты мой народ,\\
У царя ж другое мненье --\\
Он свободы не даёт.\\
\end{verse}

\uespeaks
Mordechai wollte sein Volk aus persischer Herrschaft befreien,
Achaschwerosch wollte ihn aber nicht hören.


\scene

\StageDir{Сцена пустая. Шум за сценой, звон бокалов.\\Входит \ah.}

\ahspeaks \direct{Поёт на мотив \textsc{Ой мороз мороз}}
\begin{verse}
У меня жена, ой красавица!\\
Да, красавица, всем понравится.\\
\end{verse}

\uespeaks
Achaschwerosch will seine Frau den Gästen präsentieren.

\ahspeaks
Mein Weib, Waschti, komm zu uns und tanz vor meinen Gästen!

\vspeaks
Ich bin keine Tänzerin, sondern die Zarin!

\ahspeaks \direct{Поёт на мотив \textsc{Ты ж меня обманула}}
\begin{verse}
Приказал тебе я в среду --\\
Приходи ты до обеду.\\
Все пришли тебя нема\\
Как меня ты подвела!\\
\end{verse}

\begin{verse}
\textit{Припев}\\
Ты ж меня обманула!\\
Ты ж меня подвела!\\
Пред гостями всенародно опозорила царя!\\
\end{verse}

Какой скандал!

\uespeaks
Ein Skandal! Und das Unheil hat begonnen.


\scene

\StageDir{На сцену вбегает \v, тесня Ахашвероша бюстом.}

\vspeaks \direct{Поёт на мотив \textsc{Когда б имел златые горы}}
\begin{verse}
Сулил он мне златые горы\\
И реки полные вина.\\
Ждала напрасно эти годы,\\
И чаша выпита до дна!\\
\end{verse}

\begin{verse}
Его не видим месяцами,\\
Народ гаремный говорит,\\
Что он с распутными друзьями\\
Всё наше царство разорит!\\
\end{verse}

\uespeaks
Waschti mach Vorwürfe dem Achaschwerosch, denn er hat ihre Wünsche nicht erfüllt.

\ahspeaks \direct{Поёт на мотив \textsc{Когда б имел златые горы}}
\begin{verse}
Зачем ты, Вашди, перед всеми\\
Такие вещи говоришь.\\
Ты сеешь бунт в моём гареме,\\
Эмансипацию плодишь!\\
\end{verse}

\begin{verse}
Я не позволю предо мною\\
Такие речи говорить.\\
Ты подрываешь все устои --\\
Придётся мне тебя казнить\\
Ты подрываешь все устои --\\
\direct{Приказным тоном} Палач, казнить её, казнить!\\
\end{verse}

\StageDir{Треск барабанов. \v проваливается.}

\uespeaks
Achaschwerosch war wütend und hat Waschti hinrichten lassen. In diesen früheren Zeiten
hat der Kampf der Frauen für Emanzipation sehr oft zur Hinrichtung geführt.

\scene

\StageDir{Выходит \am. В зале шум и свист.}

\uespeaks
Hamann kommt!

\ahspeaks
Was ist los?

\amspeaks
Das Volk demonstriert gegen den Bau des unterirdischen Karawanserei.
Sie meinen es ist zu teuer.

\ahspeaks
Versuche es mit demokratischen Mitteln -- kaltes Wasser und Pfefferspray.

\amspeaks
Chef, Sie sind ein Genie! \direct{хихикает}

\vdots

\amspeaks
Mein Herr, dein erster Minister wirft sich bescheiden zu deinen Füßen.
Wie du mir befohlen hast, habe ich für dein Reich eine Hymne entworfen.

\amspeaks \direct{Поёт на мотив \textsc{Боже царя храни}}
\begin{verse}
Боже царя храни\\
Царствуй любимый,\\
Царствуй всесильный\\
Во славу нам!\\
\end{verse}

\ahspeaks
Ja-ja, nicht schlecht, man braucht noch Nacharbeiten, aber immer hin. 
Mir gefält aber viel besser \direct{напевает первые четыре такта \textsc{Хава нагила}}.
Und jetzt -- das Wichtigste!

\ahspeaks \direct{Поёт на мотив \textsc{Шаланды полные кефали}}
\begin{verse}
Давай не будем тратить время,\\
Казнил я Вашди сгоряча.\\
Теперь вакансия в гареме,\\
Аман, дружище, выручай!\\
\end{verse}

\uespeaks
Achaschwerosch bat Hamann, ihm eine neue Frau zu finden.

\amspeaks \direct{Поёт на мотив припева \textsc{Смуглянка}}
\begin{verse}
Что за горе? Не печалься, дорогой!\\
Мы тя женим хоть сегодня на другой!\\
Вытри слезы, вытри слезы, мой любимый, родной!\\
А в заначке, я признАюсь не тая --\\
Есть каталог всех красоток для тебя,\\
Всех красоток, что мечтают стать женою царя!\\
\end{verse}


\direct{\ue показывает Ахашверошу силуэты кандидаток. \ah отнекивается.}

\amspeaks \direct{хихикает}
Ihr seid aber wählerisch geworden! Dann soll ich mich an ElitePartner.de wenden.
Da sind alle Schönheiten von der Vergangenheit bis in die Zukunft dargestellt --
von Eva bis Angela.


\scene

\StageDir{Выходит \e. Звучит мелодия песни \textsc{Бублички}.}

\uespeaks
Da ist unsere Esther, die schöne Adoptivtochter unseres treuen Mordechai.
Ein gut erzogenes, bescheidenes, sehr intelligentes Mädchen.

\ahspeaks \direct{Поёт на мотив \textsc{Очи чёрные}}
\begin{verse}
Очи черные, очи жгучие,\\
Очи страстные и прекрасные,\\
Как люблю я вас.\\
Как хочу я вас!\\
Подойди ко мне,\\
Этот час для нас!\\
\end{verse}

\espeaks
Was soll das? Ich bin ein anständiges Mädchen. Neulich habe ich
einen Heiratsantrag vom Herzog von Oldenburg höchstpersönlich bekommen!

\uespeaks
Und?

\espeaks
Abgelehnt! Zu klein für meine Maßstäbe. Und wer ist der?

\uespeaks
Achaschwerosch, der Herrscher von Persien.

\espeaks
Oh!

\ahspeaks \direct{Поёт влюблённно на мотив \textsc{Как много девушек хороших}}
\begin{verse}
Как много девушек хороших,\\
Как много ласковых имён,\\
Но ты одна меня пленила,\\
Я потерял покой и сон,\\
И я влюблён!\\
\end{verse}

\begin{verse}
\textit{Припев}\\
Эстер, как хорошо, что ты такая,\\
Эстер, с тобой хочу всю жизнь прожить,\\
Эстер, к твоим ногам я припадаю,\\
Готов полцарства за нежность взгляда подарить!\\
\end{verse}

\espeaks \direct{Поёт на мотив \textsc{Как много девушек хороших}}
\begin{verse}
Его признанье очень лестно,\\
Хоть он, увы, не Соломон\ldots\\
Но предложенье -- интересно,\\
И с царём персидским я взойду на трон!\\
\end{verse}

\StageDir{\ah и \e поют по очереди на мотив припева}

\speaker{А.}
Эстер, принять готова ль предложенье?

\speaker{Э.}
Прежде составим брачный мы контракт.

\speaker{А.}
Эстер, пойми, пойми моё стремленье --\\
Всё после свадьбы, всё подписать я буду рад!

\speaker{Э.} \direct{Поёт на мотив \textsc{Как много девушек хороших}}
\begin{verse}
Ах, нет, мой козлик, что я слышу!\\
У нас к царям доверья нет.\\
Контракт с раввином мы подпишем,\\
Потом -- хупа, потом -- банкет. Потом -- банкет!\\
\end{verse}

\speaker{А.}
Может, сегодня свадьба -- завтра подпись?

\speaker{Э.}
Нет уж -- всё хочешь ты наоборот.

\speaker{А.}
Может, под вечер свадьба -- утром подпись­­?

\speaker{Э.}
Конечно, можно! Но только подписи -- вперёд!

\vspace{3ex}

\uespeaks
So intelligent, wie sie war, hat Esther den Heiratsantrag nicht wiederholen lassen.

\vspace{3ex}

\espeaks \direct{мечтательно}
Hupa bestellen wir in Deutschland. Die Hupa aus Oldenburg ist die beste Hupa der Welt!

\ahspeaks
Sei bitte vernünftig! Wegen des Embargos können wir uns nur eine aus China leisten!

\espeaks
Dann mieten wir uns eine.

\ahspeaks
Übersetzer, hast du es gehört?

\uespeaks
Jawohl, Majestät, bin schon auf dem Weg!

\scene

\StageDir{Торговец приносит хупу, подаёт её рассказчику. Рассказчик устанавливает хупу.
          Торговец просит жестом чаевые.}

\uespeaks
Majestät, er möchte Trinkgeld haben.

\ahspeaks
Die Petrodollars kommen nur in großen Scheinen. Gib ihm das Geld, ich gebe es dir später\ldots
\direct{тише} oder auch nicht.

\uespeaks
Jawohl, Majestät! \direct{В сторону, со злостью} Verdammt, warum immer ich?

\StageDir{Рассказчик даёт служке купюру. Служка удаляется.}

\StageDir{Рассказчик устанавливает хупу. Жених и невеста становятся под хупу.
          У невесты в руках свиток, у жениха -- стакан. Он бросает стакан.
          Жених и невеста целуются и уходят.}

\uespeaks
Und so ist es geschehen. Die schöne jüdische Tochter Esther
wurde Frau des Herrschers des Persischen Reiches. Masel tow!


\scene

\StageDir{Входит \m. С другой стороны -- \am.}

\amspeaks \direct{Стихами}
\begin{verse}
Что стоишь, как истукан?\\
Припади к моим ногам,\\
Иль не видишь -- пред тобою\\
Сам великий князь Аман!\\
\end{verse}

\mspeaks \direct{Стихами}
\begin{verse}
Здравствуй, здравствуй князь Аман!\\
Хоть велик твой княжий сан,\\
Ты не б-г, чтобы еврею\\
Припадать к твоим ногам.\\
\end{verse}

\direct{\m уходит.}

\amspeaks
\begin{verse}
Ах, так, еврей? Ну, погоди!\\
Встреча будет впереди!\\
\end{verse}

\StageDir{\am уходит, сердясь.}

\uespeaks
So lebten sie einige Zeit. Der gute Mordechai war fleißig und gut,
und der böse Hamann war neidisch und, wie gesagt, böse. Er suchte
immer nach einer Möglichkeit, den Mordechai zu vernichten.


\scene

\StageDir{На сцену выходит \ah.}

\ahspeaks \direct{Поёт на мотив \textsc{Крутится вертится шар голубой}}
\begin{verse}
Что-то не спится мне вот уж три дня,\\
Мысли о кризисе гложат меня.\\
Их не могу я никак разрешить,\\
Нужно Амана об этом спросить.\\
\direct{Поёт на мотив \textsc{Тум балалайка}}\\
Где же Аман, где советничек мой?\\
Жду с нетерпением встречи с тобой!\\
\end{verse}

\StageDir{Входит \am.}

\ahspeaks
\begin{verse}
Вот и Аманчик, советничек мой,\\
Очень я рад нашей встрече с тобой!\\
\end{verse}

\StageDir{\am кланяется.}

\ahspeaks \direct{Поёт на мотив \textsc{Крутится вертится шар голубой}}
\begin{verse}
Что ж во дворец не приходишь, дружок,\\
Где пропадал и откуда прибёг?\\
Очень мне нужен совет верный твой:\\
В Персии кризис, развал и застой.\\
\end{verse}

\uespeaks
In dieser Zeit sind in Persien gleichzeitig eine große Dürre,
eine Plage und eine Finanzkrise aufgebrochen.

Wie beseitigen wir das Alles, mein treuer Hamann?
–- fragte Achaschwerosch seinen ersten Minister.

\amspeaks \direct{Поёт на мотив \textsc{Крутится вертится шар голубой}}
\begin{verse}
Что говоришь-то, развал и застой?\\
Как не помочь -- вот рецептик простой:\\
Я кстати пасквиль один написал,\\
Чтобы народ про виновных узнал.\\
\end{verse}

\direct{Поёт на мотив \textsc{Тум балалайка}}
\begin{verse}
Эти злодеи к Сиону сошлись\\
И всему миру вредить поклялись.\\
Кризис придумали, крах и дефолт.\\
Есть, говорят, обо всём протокол!\\
\end{verse}

\direct{Поёт на мотив \textsc{Крутится вертится шар голубой}}
\begin{verse}
Ты бы указик мне тут подписал,\\
Я бы его чёрной сотне раздал:\\
Пусть веселится, гуляет народ --\\
Мысль о кризисе враз пропадёт!\\
\end{verse}


\uespeaks
Das Wort "`beseitigen"' hat dem Hamann sehr gut gepasst. Er weiß,
was man beseitigen soll, um bei der Bevölkerung populär zu werden.
Juden! Er hat Achaschwerosch eine vollkommen gelogene Geschichte
über Zionisten erzählt, die angeblich die ganze friedliche totalitäre Welt
vernichten wollen!

\ahspeaks
So einfach?

\amspeaks
Einfach genial! \direct{хихикает}

\ahspeaks
Wie viel Zeit brauchst du für die Vorbereitung?

\amspeaks
Einen Monat, und am 14. Adar zerschlagen wir mit dem Pogrom
das internationale Zionistische Komplott!

\StageDir{\ah подписывает указ.\\ \am уходит.}

\ahspeaks \direct{ему вслед}
Ну, давай, работай, работай!


\scene

\StageDir{Появляется \m.}

\ahspeaks \direct{обращаясь к Мордехаю}
Sei gegrüßt, mein Schwager! Na, alles klar? Du, Mordechai,
Hamann hat mir heute ein Projekt vorgestellt. Heißt
"`Krisenbekämpfung im Persischen Reich"'. Wirst du Mitinvestor?

\mspeaks
Zu Euren Diensten, mein Herr. Würden Sie mich mit den Einzelheiten
des Projektes vertraut machen?

\ahspeaks
Die Idee ist einfach genial –- am 14. Adar schlagen wir die Zionisten
in unserem Reich tot, und nehmen ihren Besitz in unseren Besitz! Na, was sagst du dazu?

\StageDir{\m падает в обморок.}

\ahspeaks
In Ohnmacht gefallen vor lauter Freude!

\uespeaks
Achaschwerosch wusste nicht, dass gute Juden und Zionisten
das Gleiche sind. Was geschieht jetzt?


\scene

\StageDir{Входит \e.}

\espeaks
Mordechai, mein lieber Vater, was ist dir zugestoßen?

\mspeaks \direct{Поёт на мотив \textsc{Крутится вертится шар голубой}}
\begin{verse}
Злобный Аман хочет нам отомстить,\\
Чёрную сотню на нас напустить.\\
Хочет евреев совсем извести.\\
Как бы угрозу от нас отвести?\\
\end{verse}

\direct{Поёт на мотив \textsc{Тум балалайка}}
\begin{verse}
Ахашверошу подсунул доклад,\\
Что сионисты господства хотят.\\
Этот кретин уж указ подписал.\\
Вот и конец для евреев настал.\\
\end{verse}

\direct{Поёт на мотив \textsc{Крутится вертится шар голубой}}
\begin{verse}
Эстер, голубка, твой час наступил,\\
Действуй скорее, дай Бог тебе сил!\\
Нужно, чтоб царь отменил свой указ,\\
Чтобы не думал он плохо о нас.\\
\end{verse}
              

\uespeaks
Mordechai hat seiner klugen Tochter alles erzählt. Deine Stunde
ist gekommen, -- hat er gesagt –- bitte, klär deinen ungebildeten Mann auf,
wer die Zionisten sind. Nur du kannst unser Volk retten!

\espeaks \direct{Поёт на мотив припева \textsc{Прекрасная маркиза}}
\begin{verse}
Ах, Мордехай, какое это горе,\\
Какой кошмар, какой кошмар!\\
Сейчас с царём находимся мы в ссоре,\\
А месяц близится Адар!\\
\end{verse}

\begin{verse}
А если я к нему приду --\\
Накличу на себя беду.\\
Я не смогу помочь тогда народу,\\
Угрозы я не отведу!\\
\end{verse}


\mspeaks \direct{Поёт на мотив припева \textsc{Прекрасная маркиза}}
\begin{verse}
Ах, не проста, Эстер, твоя задача,\\
В молитве силы почерпни\\
И попостись, тогда придёт удача.\\
Остались считанные дни!\\
\end{verse}

\speaker{Э.}
Я так волнуюсь, Мордехай!

\speaker{М.}
Иди скорей, и не вздыхай.

\speaker{Э.}
Зависит от меня судьба евреев.

\speaker{М.}
Вперед, Эстер, не подкачай!


\mspeaks \direct{Поёт на мотив припева \textsc{Прекрасная маркиза}}
\begin{verse}
А вы, евреи, дома не сидите.\\
Настал ваш день, настал ваш час!\\
И в синагогу вы скорей бегите,\\
Молитесь, милые, за нас.\\
\end{verse}


\StageDir{Музыка продолжается.\\ \m и \e уходят.}

\uespeaks
Oh weh! –- hat Ester zu Mordechai gesagt, -- ich befinde mich
gerade auf Kriegsfuß mit meinem Mann. Wenn ich zu ihm gehe,
verliere ich meinen Kopf, dann bin ich für unser Volk absolut unnütz!

Lies die Tora, -- hat Mordechai Ester gesagt, -- und Fasten
schadet auch nicht. Und ihr, Juden, betet auch mit, denn die Zeit läuft.


\scene

\StageDir{Царские покои.\\ \ah.\\Входит \e.}

\espeaks \direct{Поёт на мотив \textsc{Крутится вертится шар голубой}}
\begin{verse}
Милый, по делу к тебе я пришла,\\
Страшную весть я тебе принесла.\\
Выслушай, милый, казнить не вели:\\
Чёрные силы интриги сплели.\\
\end{verse}

\ahspeaks \direct{Поёт на мотив \textsc{Крутится вертится шар голубой}}
\begin{verse}
Как хорошо, что пришла ты, жена.\\
Я третий день всё терзаюсь без сна.\\
Видно, я глупость опять совершил.\\
Где-то, наверное, я нагрешил.\\
\end{verse}


\uespeaks
Gott war gnädig zu Ester. Als sie zu Achaschwerosch kam, hat er sie
bereits erwartet. Hilf mir, meine Liebe, -- hat er gesagt, -- ich habe letzte Zeit Alpträume.

\StageDir{Бой барабанов.}

\ahspeaks \direct{Поёт на мотив \textsc{Цыганской песни "<В цыганах закипела кровь">} из оперы \textsc{Кармен}}
\begin{verse}
Эстер, я видел страшный сон:\\
Вороны надо мной летали,\\
И больно в голову клевали,\\
И гадили на царский трон.\\
\end{verse}

\begin{verse}
Среди ворон кружит Аман,\\
И убежать к тебе мешает.\\
Мой трон помётом украшает,\\
А я сижу, как истукан\ldots\\
\end{verse}

\begin{verse}
Эстер, ты -- мудрая жена:\\
В чём тайный смысл такого сна?\\
\end{verse}


\uespeaks
In meinem Alptraum sitze ich auf meinem Herrscherthron, und Hamann fliegt
mit den Krähen über meinem Kopf, und er kackt auf mich. Was kann das bedeuten?

\espeaks \direct{Поёт на мотив \textsc{Цыганской песни "<В цыганах закипела кровь">} из оперы \textsc{Кармен}}
\begin{verse}
Так знай же, царь, тот вещий сон\\
Не просто так тебе приснился.\\
Тебя предупреждает он,\\
Чтоб ты опять не оступился.\\
\end{verse}

\begin{verse}
Коварную фальшивку\\
Аман тебе подсунул,\\
А ты схватил наживку.\\
О чём тогда ты только думал?\\
\end{verse}

\begin{verse}
Собрав отребье с миру,\\
Устроив здесь нам всем погром,\\
Используя их силу,\\
Взойдёт Аман на трон.\\
\end{verse}

\begin{verse}
Евреи у тебя в стране --\\
Все сплошь врачи и музыканты,\\
Приносят прибыли казне.\\
Используй же ты их таланты!\\
\end{verse}

\begin{verse}
Ты разрешал им до сих пор\\
Трудиться мирно, честно.\\
И вдруг готовишь приговор\\
Убить их повсеместно!\\
\end{verse}


\uespeaks
"`Oh, mein Liebster, -- hat Ester zu ihm gesagt, -- das kann nur eines bedeuten:
Dass du in große Scheiße geraten bist."' Und sie hat ihrem Mann alles erklärt:
wer Zionisten sind, und wer Juden, und was für gute Menschen sie alle sind,
und hat ihm Mordechais Verdienste wieder aufgelistet.


\espeaks
\begin{verse}
Не мне беседу заводить\\
О том, кто заговор раскрыл.\\
Его хотел ты наградить,\\
Но, как всегда, забыл.\\
\end{verse}

\begin{verse}
Ты знай: как только сатана\\
Тебе указ подложит --\\
Погибнет и твоя жена,\\
Ведь я -- еврейка тоже!\\
\end{verse}

\begin{verse}
Кем станешь ты -- губителем\\
Ста тысяч человек,\\
Или царём-спасителем\\
Сегодня -- и навек?\\
\end{verse}


\uespeaks
Natürlich hat sie den bösen Hamann auch erwähnt, und seine
"`Verdienste"' in Anführungsstrichen auch\ldots

\StageDir{\e уходит.}


\scene

\StageDir{Входит \am.}

\amspeaks \direct{Поёт на мотив \textsc{Когда б имел златые горы}}
\begin{verse}
Ты что сидишь мрачнее тучи?\\
Опять случилась, что ль, беда?\\
Смотри -- погромщиков докучи!\\
Готовы к бою мы всегда!\\
\end{verse}

\begin{verse}
Я разгромлю евреев стаи,\\
Очищу землю от жидов.\\
Ты награди меня медалью.\\
Награду я принять готов!\\
\end{verse}


\ahspeaks \direct{Поёт на мотив \textsc{Marseillaise}}
\begin{verse}
Ты хочешь от меня награды\\
За то, что сам тут натворил.\\
Собрал ты чёрные бригады,\\
Народ к разбою ты подбил!\\
\end{verse}

\begin{verse}
Прислал ООН протеста ноту.\\
Европа собирает рать,\\
Чтоб за твою, Аман, работу\\
Всех персов строго наказать.\\
\end{verse}

\begin{verse}
За то, что ты -- такой вредитель,\\
Отдам тебя я палачу.\\
А Мордыхай -- он мой спаситель.\\
Его я наградить хочу!\\
\end{verse}


\uespeaks
Als Hamann zu Achaschwerosch am 14. Adar kam, von Tatendrang  erfüllt,
hat Achaschwerosch ihn gefragt, was er als erster Minister machen würde,
wenn Jemand für sein Land etwas außerordentlich Gutes täte. Hamann dachte,
dass es sich um ihn handelte, und hat gesagt, dass man den gut belohnen soll.
\par Und was man mit einem Verbrecher tun muss? Da dachte Hamann, dass es sich
um Mordechai handelt, und sagte, dass man ihn hinrichten soll.


\scene

\StageDir{Выезжает \m на белом коне с израильским флагом.}

\mspeaks
\begin{verse}
Пришли другие времена.\\
Назло Аманским стаям\\
Живёт народ, цветёт страна,\\
И Пурим мы справляем!\\
Израиль  вечно пусть живёт!\\
Мы скажем все "<лехаим">\\
За весь еврейский наш народ:\\
Лехаим! Лехаим! Лехаим!\\
\end{verse}


\uespeaks
Und so geschah es. Der Verbrecher Hamann wurde hingerichtet,
der gute Mordechai wurde belohnt mit seinem ganzen Volk. 
Wir feiern heute Purim, wir lachen, singen, Trinken und sagen: Lechaim!

\StageDir{Занавес закрывается. Звучит песня "<Лехаим">.\\Актёры выходят на поклон.}

\end{drama}


\end{document}
